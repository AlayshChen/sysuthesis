% vim:ts=4:sw=4
%
% Copyright (c) 2008-2009 solvethis
% Copyright (c) 2010-2012 Casper Ti. Vector
% Public domain.
%
% 使用前请先仔细阅读 pkuthss 和 biblatex-caspervector 的文档,
% 特别是其中的 FAQ 部分和用红色强调的部分。
% 两者可在终端/命令提示符中用
%   texdoc pkuthss
%   texdoc biblatex-caspervector
% 调出。

% 在黑白打印时彩色链接可能变成浅灰色,此时可将“colorlinks”改为“nocolorlinks”。
\documentclass[UTF8, colorlinks, notocbibind]{pkuthss}

% 使用 biblatex 排版参考文献,并规定其格式。
\usepackage[backend = biber, style = caspervector, utf8]{biblatex}
% 使得打字机粗体可以被使用。
\usepackage{lmodern}
% 产生 originauth.tex 里的 \square。
\usepackage{amssymb}
% 提供 Verbatim 环境和 \VerbatimInput 命令。
\usepackage{fancyvrb}

% 使被强调的内容为红色。
\newcommand{\myemph}[1]{\emph{\textcolor{red}{#1}}}

% pkuthss 文档模版的版本。
\newcommand{\docversion}{v1.4 rc1}
% 设定文档的基本信息。
\pkuthssinfo{
	cthesisname = {硕士学位论文}, ethesisname = {Sun Yat-sen University Master Thesis},
	ctitle = {可信安全操作系统中访问控制机制的研究与实现},
	% “\\”在设定 pdf 属性时会被自动过滤掉,于是得到的 pdf 属性中标题为
	%   The PKU dissertation document classpkuthss [版本号]
	% 此处指定其被替换为“: ”,以使之为
	%   The PKU dissertation document class: pkuthss [版本号]
	etitle = {%
       Research and Implementation of Access Control Mechanism\\ in Trusted Security Operating System
	},
	cauthor = {罗尼 $\cdot$ 奥沙利文},
	eauthor = {Ronnie O'Sullivan},
	studentid = {07302???},
	date = {二〇一一年五月},
	school = {信息科学与技术学院},
	cmajor = {计算机技术}, emajor = {Computer Technology},
	direction = {强力斯诺克},
	cmentor = {XX 教授}, ementor = {Prof.\ XX},
	ckeywords = {\LaTeX2e{},排版,文档类,\CTeX{}},
	ekeywords = {\LaTeX2e{}, typesetting, document class, \CTeX{}}
}
% 导入参考文献数据库(注意不要省略“.bib”)。
\addbibresource{sample.bib}

\begin{document}
	% 以下为正文之前的部分。
	\frontmatter

	% 自动生成标题页。
    \newgeometry{height = 220mm, width = 160mm }
	\maketitle
    \restoregeometry
	% 版权声明。
	\cleardoublepage
\chapter*{\textbf{版权声明}}
{
	\zihao{3}
\begin{comment}
	任何收存和保管本论文各种版本的单位和个人,
	未经本论文作者同意,不得将本论文转借他人,
	亦不得随意复制、抄录、拍照或以任何方式传播。
	否则一旦引起有碍作者著作权之问题,将可能承担法律责任。
\end{comment}
	版权所有~\copyright~1991-2010 Casper Ti. Vector

	本文档可在~GNU~自由文档许可证(GFDL)\footnote%
	{\ \url{http://www.fsf.org/licensing/licenses/fdl.html}}%
	的第~1.3~版(或之后任意版本)或~GNU~通用公共许可证(GPL)\footnote%
	{\ \url{http://www.fsf.org/licensing/licenses/gpl.html}}%
	的第~3~版(或之后任意版本)所规定的条款下自由地复制、修改和发布。
	
	以上所述两个许可证应该在本文档所在目录的~\verb|license/|~%
	\linebreak[1]子目录下,
	文件名分别为~\verb|fdl-1.3.txt|~和~\verb|gpl-3.0.txt|。
	如果没有,你可以到上面提到的网址查看许可证内容。
	如果还不行,请写信给下面的地址以获得邮寄的许可证:
\begin{verbatim}
    The Free Software Foundation, Inc.,
    675 Mass Ave, Cambridge, MA02139, USA 
\end{verbatim}
	\par
}


	% 原创性声明和使用授权说明。
	% vim:ts=4:sw=4
%
% Copyright (c) 2008-2009 solvethis
% Copyright (c) 2010-2012 Casper Ti. Vector
% All rights reserved.
%
% Redistribution and use in source and binary forms, with or without
% modification, are permitted provided that the following conditions are
% met:
%
% * Redistributions of source code must retain the above copyright notice,
%   this list of conditions and the following disclaimer.
% * Redistributions in binary form must reproduce the above copyright
%   notice, this list of conditions and the following disclaimer in the
%   documentation and/or other materials provided with the distribution.
% * Neither the name of Peking University nor the names of its contributors
%   may be used to endorse or promote products derived from this software
%   without specific prior written permission.
% 
% THIS SOFTWARE IS PROVIDED BY THE COPYRIGHT HOLDERS AND CONTRIBUTORS "AS
% IS" AND ANY EXPRESS OR IMPLIED WARRANTIES, INCLUDING, BUT NOT LIMITED TO,
% THE IMPLIED WARRANTIES OF MERCHANTABILITY AND FITNESS FOR A PARTICULAR
% PURPOSE ARE DISCLAIMED. IN NO EVENT SHALL THE COPYRIGHT HOLDER OR
% CONTRIBUTORS BE LIABLE FOR ANY DIRECT, INDIRECT, INCIDENTAL, SPECIAL,
% EXEMPLARY, OR CONSEQUENTIAL DAMAGES (INCLUDING, BUT NOT LIMITED TO,
% PROCUREMENT OF SUBSTITUTE GOODS OR SERVICES; LOSS OF USE, DATA, OR
% PROFITS; OR BUSINESS INTERRUPTION) HOWEVER CAUSED AND ON ANY THEORY OF
% LIABILITY, WHETHER IN CONTRACT, STRICT LIABILITY, OR TORT (INCLUDING
% NEGLIGENCE OR OTHERWISE) ARISING IN ANY WAY OUT OF THE USE OF THIS
% SOFTWARE, EVEN IF ADVISED OF THE POSSIBILITY OF SUCH DAMAGE.

% 原创性声明和使用授权说明页不需要装订到论文中,故不显示页码。
\cleardoublepage\thispagestyle{empty}
{
	\linespread{1.5}\selectfont
	\section*{原创性声明}

	本人郑重声明:
	所呈交的学位论文,是本人在导师的指导下,独立进行研究工作所取得的成果。
	除文中已经注明引用的内容外,
	本论文不含任何其他个人或集体已经发表或撰写过的作品或成果。
	对本文的研究做出重要贡献的个人和集体,均已在文中以明确方式标明。
	本声明的法律结果由本人承担。
	\vspace{2.5em}\par
	\rightline
	{%
		论文作者签名:\hspace{5em}%
		日期:\hspace{2em}年\hspace{2em}月\hspace{2em}日%
	}
    \par
}


	% 中英文摘要。
	% 摘要要求在 300~500字 。

\cleardoublepage
\begin{cabstract}

	这个模板是从~\textit{pkuthss}~修改而来,旨在维护一个方便易用的论文模板。
\end{cabstract}

\cleardoublepage
\begin{eabstract}
	
	This template is modified from \textit{pkuthss} \ldots 
\end{eabstract}


	% 自动生成目录。
	\tableofcontents

	% 以下为正文。
	\mainmatter

	% 绪言。
	% vim:ts=4:sw=4
%
% Copyright (c) 2008-2009 solvethis
% Copyright (c) 2010-2012 Casper Ti. Vector
% Public domain.

\specialchap{绪言}

本文档是“北京大学论文文档模版”的说明文档,
同时也是使用模版的一个示例。

pkuthss 文档模版由三部分构成:
\begin{itemize}
	\item \textbf{pkuthss 文档类}:
		其中进行了学位论文所需要的一些基本的设定,
		主要包括对基本排版格式的设定和提供设置论文信息的命令。
	\item \textbf{pkuthss-extra 宏包}:
		其中实现了学位论文中用户可能较多用到的一些额外功能,
		例如自动在目录中加入参考文献和索引的条目和%
		自动根据用户设定的文档信息对所生成 pdf 的作者、标题等属性进行设置等。
	\item \textbf{说明(示例)文档}:
		说明文档即本文档,
		在安装(见第 \ref{sec:inst} 节)之后应该可以用 \TeX{} 系统提供的
		\verb|texdoc| 命令调出:
\begin{Verbatim}[frame=single]
texdoc pkuthss
\end{Verbatim}
		同时,
		本文档的源代码(位和本文档的 pdf 文件处于同一目录下)%
		也正是用户撰写自己的学位论文时的一个模版:
		用户只需按照模版中的框架修改代码,
		即可写出自己的论文。
\end{itemize}

在此之前,包括 dypang\supercite{dypang}、FerretL\supercite{FerretL}、%
lwolf\supercite{lwolf}、Langpku\supercite{Langpku}、%
solvethis\supercite{solvethis} 等的数位网友均做过学位论文模版的工作。
本论文模版是 solvethis 的 pkuthss 模版的更新版本,
更新的重点是重构和对新文档类、宏包的支持。

pkuthss 文档模版现在的维护者是 Casper Ti. Vector\footnote%
{\href{mailto:CasperVector@gmail.com}{\texttt{CasperVector@gmail.com}}}。%
pkuthss 文档模版目前托管在 Google Code 上,
其项目主页是:\\
\hspace*{\parindent}\url{http://code.google.com/p/caspervector/}


	% 各章节。
	\chapter{第一章标题}
\section{测试}
测试引用\supercite{AC1}\supercite{AC4}\supercite{SEL3}

	% vim:ts=4:sw=4
%
% Copyright (c) 2008-2009 solvethis
% Copyright (c) 2010-2012 Casper Ti. Vector
% Public domain.

\chapter{pkuthss 文档模版提供的功能}
	\section{pkuthss 文档模版提供的文档类和宏包选项}
		\subsection{pkuthss 文档类提供的选项}\label{ssec:options}

		\begin{itemize}
			\item \textbf{\texttt{[no]extra}}:
				用于确定是否自动载入 pkuthss-extra 宏包。
				在默认情况下,pkuthss 文档类将使用 \verb|extra| 选项。
				用户如果不需要自动载入 pkuthss-extra 宏包,
				则需要在载入 pkuthss 时加上 \verb|noextra| 选项。

			\item \textbf{pkuthss-extra 宏包提供的选项}:
				这些选项将被传递给 pkuthss-extra 宏包
				(用户需要启用 \verb|extra| 选项)。
				具体说明参见第 \ref{ssec:extra} 小节。

			\item \textbf{其余文档类选项}:%
				pkuthss 文档类以 ctexbook 文档类为基础,
				其接受的其余所有文档类选项均被传递给 ctexbook。
				其中可能最常用的选项是 \verb|GBK| 和 \verb|UTF8|:
				它们选择源代码使用的字符编码,默认使用 \verb|GBK|。
		\end{itemize}

		例如,如果需要使用 UTF-8 编码撰写论文,
		则需要在导入 pkuthss 文档类时加上 \verb|UTF8| 选项:
\begin{Verbatim}[frame = single]
\documentclass[UTF8, ...]{pkuthss} % “...”代表其它的选项。
\end{Verbatim}

		又例如,文档默认情况下是双面模式,每章都从右页(奇数页)开始。
		如果希望改成一章可以从任意页开始,可以这样设置:
\begin{Verbatim}[frame = single]
\documentclass[openany, ...]{pkuthss} % 每章从任意页开始。
\end{Verbatim}
		但这样设置时左右(奇偶)页的页眉页脚设置仍然是不同的。
		如果需要使左右页的页眉页脚设置一致,可以直接采用单面模式:
\begin{Verbatim}[frame = single]
% 使用 oneside 选项时不需要再指定 openany 选项。
\documentclass[oneside, ...]{pkuthss}
\end{Verbatim}

		\subsection{pkuthss-extra 宏包提供的选项}\label{ssec:extra}

		除非特别说明,
		下面提到的选项中都是不带“\verb|no|”的版本被启用。

		\begin{itemize}
			\item \textbf{\texttt{[no]spacing}}:
				是否采用一些常用的对空白进行调整的版式设定。
				具体地说,启用 \verb|space| 选项后会进行以下几项设置:
			\begin{itemize}
				\item 自动忽略 CJK 文字之间的空白而%
					保留(CJK 文字与英文之间等的)其它空白。
				\item 调用 setspace 宏包以使某些细节处的空间安排更美观。
				\item 设置页芯居中。
				\item 设定行距为 1.41\footnote{%
					为什么是 1.41?因为 $\sqrt{2}\approx1.41$。%
				}。
				\item 使脚注编号和脚注文本之间默认间隔一个空格。
			\end{itemize}

			\item \textbf{\texttt{[no]tightlist}}:
				是否采用比 \LaTeX{} 默认设定更加紧密的枚举环境。
				在枚举环境(itemize、enumerate 和 description)中,
				每个条目的内容较少时,条目往往显得稀疏;
				在参考文献列表中也有类似的现象。
				启用 \verb|tightlist| 选项后,
				将去掉这些环境中额外增加的(垂直)间隔。

			\item \textbf{\texttt{[no]pdftoc}}\footnote{%
					此选项部分等价于 1.4 alpha2 及以前版本 pkuthss-extra 宏包%
					的 \texttt{[no]tocbibind} 选项。
					因为 tocbibind 宏包和 biblatex 宏包冲突,
					pkuthss-extra 宏包不再调用 tocbibind 宏包。%
				}:
				启用 \verb|pdftoc| 选项后,
				用 \verb|\tableofcontents| 命令生成目录时%
				会自动添加“目录”的 pdf 书签。

			\item \textbf{\texttt{[no]uppermark}}:
				是否在页眉中将章节名中的小写字母转换为大写字母。
				就目前而言,
				这样的转换存在着一些较为严重的缺陷\footnote{%
					准确地说是 \texttt{\string\MakeUppercase} 宏的问题,
					其在某些地方的转换不够健壮,
					例如 \texttt{\string\cite\string{ctex\string}}
					会被转换成 \texttt{\string\cite\string{CTEX\string}}。%
				},
				因此不建议使用。
				基于上述考虑,%
				\myemph{%
					pkuthss-extra 宏包默认启用 \texttt{nouppermark} 选项,
					即在不在页眉中使用大写的章节名%
				}。

			\item \textbf{\texttt{[no]spechap}}\footnote{%
					“spechap”是“\textbf{spec}ial \textbf{chap}ter”的缩写。%
				}:
				是否启用第 \ref{ssec:misc} 小节中介绍的 %
				\verb|\specialchap| 命令。

			\item \textbf{\texttt{[no]pdfprop}}:
				是否自动根据设定的论文文档信息(如作者、标题等)
				设置生成的 pdf 文档的相应属性。%
				\myemph{%
					注意:
					该选项实际上是在 \texttt{\string\maketitle} 时生效的,
					这是因为考虑到%
					通常用户在调用 \texttt{\string\maketitle} 前%
					已经设置好所有的文档信息。
					若用户不调用 \texttt{\string\maketitle},
					则需在设定完文档信息之后自行调用%
					第 \ref{ssec:misc} 小节中介绍的 %
					\texttt{\string\setpdfproperties} 命令以完成%
					pdf 文档属性的设定。%
				}

			\item \textbf{\texttt{[no]colorlinks}}\footnote{%
					此选项等价于 1.3 及以前版本 pkuthss-extra 宏包%
					的 \texttt{[no]linkcolor} 选项,
					但后来发现这会和 hyperref 宏包的一个同名选项冲突,
					故改为 \texttt{[no]colorlinks}。%
				}:
				是否在生成的 pdf 文档中使用彩色的链接。
		\end{itemize}

		例如,在提交打印版的论文时,
		彩色的链接文字在黑白打印出来之后可能颜色会很浅。
		此时用户\myemph{%
			可以启用 pkuthss-extra 宏包的 nocolorlinks 选项,
			使所有的链接变为黑色,以免影响打印%
		}:
\begin{Verbatim}[frame = single]
\documentclass[..., nocolorlinks]{pkuthss} % “...”代表其它的选项。
\end{Verbatim}
		用户还可以进一步修改 \verb|img/| 目录中 %
		\verb|pkulogo.eps| 和 \verb|pkuword.eps| 两个图片文件,
		以使封面上的北京大学图样也变为黑色(详见相应文件中的注释)。

	\section{pkuthss 文档模版提供的命令和环境}
		\subsection{设定文档信息的命令}

		这一类命令的语法为
\begin{Verbatim}[frame = single]
\commandname{具体信息} % commandname 为具体命令的名称。
\end{Verbatim}

		这些命令总结如下:
		\begin{itemize}
			\item \texttt{\bfseries\string\ctitle}:设定论文中文标题;
			\item \texttt{\bfseries\string\etitle}:设定论文英文标题;
			\item \texttt{\bfseries\string\cauthor}:设定作者的中文名;
			\item \texttt{\bfseries\string\eauthor}:设定作者的英文名;
			\item \texttt{\bfseries\string\studentid}:设定作者的学号;
			\item \texttt{\bfseries\string\date}:设定日期;
			\item \texttt{\bfseries\string\school}:设定作者的学院名;
			\item \texttt{\bfseries\string\cmajor}:设定作者专业的中文名;
			\item \texttt{\bfseries\string\emajor}:设定作者专业的英文名;
			\item \texttt{\bfseries\string\direction}:设定作者的研究方向;
			\item \texttt{\bfseries\string\cmentor}:设定导师的中文名;
			\item \texttt{\bfseries\string\ementor}:设定导师的英文名;
			\item \texttt{\bfseries\string\ckeywords}:设定中文关键词;
			\item \texttt{\bfseries\string\ekeywords}:设定英文关键词。
		\end{itemize}

		例如,如果要设定专业为“化学”(“Chemistry”),则可以使用以下命令:
\begin{Verbatim}[frame = single]
\cmajor{化学}
\emajor{Chemistry}
\end{Verbatim}

		\subsection{自身存储文档信息的命令}

		这一类命令的语法为
\begin{Verbatim}[frame = single]
% commandname 为具体的命令名。
\renewcommand{\commandname}{具体信息}
\end{Verbatim}

		这些命令总结如下:
		\begin{itemize}
			\item \texttt{\bfseries\string\cuniversity}:大学的中文名。
			\item \texttt{\bfseries\string\euniversity}:大学的英文名。
			\item \texttt{\bfseries\string\cthesisname}:论文类别的中文名。
			\item \texttt{\bfseries\string\ethesisname}:论文类别的英文名。
			\item \texttt{\bfseries\string\cabstractname}:摘要的中文标题。
			\item \texttt{\bfseries\string\eabstractname}:摘要的英文标题。
		\end{itemize}

		例如,
		如果要设定论文的类别为“本科生毕业论文”(“Undergraduate Thesis”),
		则可以使用以下命令:
\begin{Verbatim}[frame = single]
\renewcommand{\cthesisname}{本科生毕业论文}
\renewcommand{\ethesisname}{Undergraduate Thesis}
\end{Verbatim}

		\subsection{以“key = value”格式设置文档信息}

		用户可以通过 \verb|\pkuthssinfo| 命令集中设定文档信息,
		其语法为:
\begin{Verbatim}[frame = single]
% key1、key2、value1、value2 等为具体文档信息的项目名和内容。
\pkuthssinfo{key1 = value1, key2 = value2, ...}
\end{Verbatim}
		其中文档信息的项目名为前面提到的设定文档信息的命令名%
		或自身存储文档信息的命令名(不带反斜杠)。

		当文档信息的内容包含了逗号等有干扰的字符时,
		可以用大括号将这一项文档信息的全部内容括起来。%
		\myemph{%
			我们推荐用户总用大括号将文档信息的内容括起来,
			以避免很多不必要的麻烦。%
		}

		例如,前面提到的文档信息的设置可以集中地写成:
\begin{Verbatim}[frame = single, tabsize = 4]
\pkuthssinfo{
	..., % “...”代表其它的设定。
	cthesisname = {本科生毕业论文},
	ethesisname = {Undergraduate Thesis},
	cmajor = {化学}, emajor = {Chemistry}
}
\end{Verbatim}

		\subsection{pkuthss 文档模版提供的其它命令和环境\label{ssec:misc}}

		\texttt{\bfseries cabstract} 和 \texttt{\bfseries eabstract} %
		环境用于编写中英文摘要。
		用户只需要写摘要的正文;标题、作者、导师、专业等部分会自动生成。

		\texttt{\bfseries\string\specialchap} 命令%
		用于开始不进行标号但计入目录的一章,
		并合理安排其页眉。%
		\myemph{%
			注意:
			需要启用 pkuthss-extra 宏包的 \texttt{spechap} 选项%
			才能使用此命令。
			另外,在此章内的节或小节等命令应使用带星号的版本,
			例如 \texttt{\string\section\string*} 等,
			以免造成章节编号混乱。%
		}%
		例如,本文档中的“绪言”一章就是用 \verb|\specialchap{绪言}| %
		这条命令开始的。%

		\texttt{\bfseries\string\setpdfproperties} 命令%
		用于根据用户设定的文档信息自动设定生成的 pdf 文档的属性。
		此命令会在用户调用 \verb|\maketitle| 命令时被自动调用,
		因此通常不需要用户自己使用;
		但用户有时可能不需要输出标题页,
		从而不会调用 \verb|\maketitle| 命令,
		此时就需要在设定完文档信息之后调用 \verb|\setpdfproperties|。
		\myemph{%
			注意:
			需要启用 pkuthss-extra 宏包的 \texttt{pdfprop} 选项%
			才能使用此命令。%
		}

		\subsection{从其它文档类和宏包继承的功能}

		pkuthss 文档类建立在 ctexbook\supercite{ctex} 文档类的基础上,
		并调用了 CJKfntef、%
		graphicx\supercite{graphicx}、geometry\supercite{geometry}、%
		fancyhdr\supercite{fancyhdr} 和 %
		etoolbox\supercite{etoolbox} 等几个宏包。
		因此,ctexbook 文档类和这些宏包所提供的功能均可以使用。

		例如,用户如果想将目录的标题改为“目{\quad\quad}录”,
		则可以使用 ctexbook 文档类提供的 \verb|\CTEXoptions| 命令:
\begin{Verbatim}[frame = single]
\CTEXoptions{contentsname = {目{\quad\quad}录}}
\end{Verbatim}

		在默认的配置下,%
		pkuthss 文档模版使用作者编写的 %
		biblatex\supercite{biblatex} 样式\supercite{biblatex-caspervector}%
		进行参考文献和引用的排版,
		用户可以使用其提供的功能。
		例如,
		用户可以分别使用 \verb|\cite|、\verb|\parencite| 和 \verb|\supercite| %
		生成未格式化的、带方括号的和上标且带方括号的引用标记:
\begin{Verbatim}[frame = single]
\cite{ctex},\parencite{ctex},\supercite{ctex}
\end{Verbatim}
		在本文中将产生“\cite{ctex},\parencite{ctex},\supercite{ctex}”。

		pkuthss-extra 宏包可能调用以下这些宏包:
		\begin{itemize}
			\item 启用 \verb|spacing| 选项时会调用 %
				setspace 和 footmisc\supercite{footmisc} 宏包。
			\item 启用 \verb|tightlist| 选项时会调用 %
				enumitem\supercite{enumitem} 宏包。
		\end{itemize}
		因此在启用相应选项时,用户可以使用对应宏包所提供的功能。

		\subsection{不建议更改的设置}
		\myemph{%
			pkuthss 文档类中有一些一旦改动就有可能破坏预设排版规划的设置,
			因此不建议更改这些设置,它们是:
			\begin{itemize}
				\item 纸张类型:A4;
				\item 页芯尺寸:%
					$240\,\mathrm{mm}\times150\,\mathrm{mm}$,
					包含页眉、页脚;
				\item 默认字号:小四号。
			\end{itemize}%
		}

	\section{高级设置}\label{sec:advanced}

	pkuthss 文档模版的实现是简洁、清晰、灵活的。
	当一些细节的自定义无法通过模版提供的外部接口实现时,
	我们鼓励用户(在适当理解相关部分代码的前提下)通过修改模版进行自定义。

	一个常见的需求是封面中部分内容(特别是论文的标题、专业和研究方向)太长,
	超出了在预设的空间。
	此时,
	用户可以修改 \verb|pkuthss.cls| 里 \verb|\maketitle| 定义中
	\verb|\pkuthss@int@fillinblank| 宏的参数来改变
	带下划线的空白的行数和行宽,其语法为:
\begin{Verbatim}[frame = single]
\pkuthss@int@fillinblank{行数}{行宽}{内容}
\end{Verbatim}
	例如,如果“研究方向”一栏需要两行的空白,
	可以将 \verb|pkuthss.cls| 里的
\begin{Verbatim}[frame = single]
\pkuthss@int@fillinblank{1}{\pkuthss@tmp@len}{\kaishu\@direction}
\end{Verbatim}
	改为
\begin{Verbatim}[frame = single]
\pkuthss@int@fillinblank{2}{\pkuthss@tmp@len}{\kaishu\@direction}
\end{Verbatim}
	当然,为了美观,可以将多于一行的部分移到封面中作者信息部分的最下方。


	% vim:ts=4:sw=4
%
% Copyright (c) 2008-2009 solvethis
% Copyright (c) 2010-2012 Casper Ti. Vector
% Public domain.

\chapter{问题及其解决}
	\section{文档中已经提到的常见问题(按重要性排序)}

	在默认设置(启用 \verb|colorlinks| 选项)下,
	黑白打印时文档中的部分彩色链接可能会变成浅灰色,
	解决方式见第 \ref{ssec:extra} 小节。

	中文字体字库不全(只包含 GB2312 字符集内字符)时,
	生成的 pdf 文档中可能缺少部分字符,
	解决方式见第 \ref{sec:req} 节。

	\verb|img/| 目录中 eps 图片未转换为 pdf 格式时,%
	pdf\LaTeX{} 方式编译可能出错,
	解决方式见第 \ref{sec:doc-dir} 节。

	使用过旧的 \TeX{} 系统和各宏包,
	或使用某些 Linux 发行版软件仓库所提供的 \TeX{}Live 时,
	可能引起一些问题,
	详见第 \ref{sec:req} 节。

	文档默认情况下是双面模式,章末可能产生空白页,详见第 \ref{ssec:options} 小节。

	一些高级设置,
	如封面中部分内容长度超过预设空间容量时的设置,
	见第 \ref{sec:advanced} 节。

	biblatex 宏包\cite{biblatex}会自行设定 \verb|\bibname|,
	故会覆盖通过 \verb|\CTEXoptions| 设定的参考文献列表标题。
	使用 biblatex 的用户可以使用 \verb|\printbibliography| 的
	\verb|title| 选项来手动设定参考文献列表的标题,例如:
\begin{Verbatim}[frame = single]
\printbibliography[title = {文献}, ...] % “...”为其它选项。
\end{Verbatim}

	\section{其它可能存在的问题}
		\subsection{上游宏包可能引起的问题}

		hyperref 宏包\supercite{hyperref}和一些宏包可能发生冲突。
		关于如何避免这些冲突,可以参考 hyperref 宏包的文档。
		此文件通常和执行 \verb|texdoc hyperref| %
		时打开的 pdf 文件位于同一目录中。
		低于 1.02c 版本的 ctex 宏包中对 hyperref 的设置有些不周,
		因此文档类中对其进行了一些手动的处理。
		考虑到新版本 ctex 宏包将逐渐被更多人采用,
		进行这些处理的代码将在以后被删除,
		而改成直接调用 ctex 宏包的 \verb|hyperref| 选项。

		\subsection{文档格式可能存在的问题}

		研究生手册和其电子版\supercite{pku-thesisstyle}要求的论文封面并不一致。
		这里以电子版为准。

		\subsection{其它一些问题}

		使用 GBK 编码和 pdf\LaTeX{} 编译方式时需要用户%
		运行 \verb|gbk2uni| 程序来转换 \verb|.out| 文件,
		否则生成的 pdf 书签可能乱码。
		考虑到用户可能没有 \verb|gbk2uni| 程序,且有用户使用 UTF-8 编码,
		默认的 \verb|Makefile| 和 \verb|Make.bat| 中将相关代码注释掉了,
		用户可以自行去掉相应的注释。

	\section{反馈意见和建议}

	关于 pkuthss 文档模版的意见和建议,
	请在北大未名 BBS 的 MathTools 版或 %
	Google Code 上 pkuthss 项目的 issue tracker%
	\footnote{\url{http://code.google.com/p/caspervector/issues/list}.}%
	上提出,
	或通过电子邮件\footnote%
	{\href{mailto:CasperVector@gmail.com}{\texttt{CasperVector@gmail.com}}.}%
	告知 Casper Ti. Vector。
	上述三种反馈方法中,建议用户尽量采用靠前的方法。

	在进行反馈时,请尽量确保已经仔细阅读本文档中的说明。
	如果是通过 BBS 或电子邮件进行反馈,
	请在标题中说明是关于 pkuthss 文档模版的反馈;
	如果是通过 Google Code 进行反馈,
	请给 issue 加上 \verb|Proj-Pkuthss| 标签。
	如果是错误报告,
	请说明所使用 pkuthss 模版的版本、
	自己使用的操作系统和 \TeX{} 系统的类型和版本;
	同时强烈建议附上一个出错的最小例子及其相应的编译日志(\verb|.log| 文件),
	在文件较长时请使用附件。


	% 结论。
	% vim:ts=4:sw=4
%
% Copyright (c) 2008-2009 solvethis
% Copyright (c) 2010-2012 Casper Ti. Vector
% Public domain.

\specialchap{结论}

pkuthss 文档模版结构较为简洁、清晰、灵活,较为易于学习和使用。
希望它能为各位需要使用 \LaTeX{} 撰写学位论文的同学提供一些帮助。



	% 正文中的附录部分。
	\appendix
	% 排版参考文献列表,并使其出现在目录中。
	% 如果同时要使参考文献列表参与章节编号,可将“bibintoc”改为“bibnumbered”。
	\printbibliography[heading = bibintoc]
	% 各附录。
	% vim:ts=4:sw=4
%
% Copyright (c) 2008-2009 solvethis
% Copyright (c) 2010-2012 Casper Ti. Vector
% Public domain.

\chapter{pkuthss 文档模版的实现}
\raggedbottom % 避免某些奇怪的“Underfull \vbox”警告。

	\section{pkuthss 文档类和 pkuthss-extra 宏包的实现}	
		\subsection{共用文件头部}
		\VerbatimInput[
			frame = lines, fontsize = {\footnotesize}, tabsize = 2,
			baselinestretch = 1, lastline = 23, numbers = left
		]{pkuthss.cls}

		\subsection{\texttt{pkuthss.cls}}
		\VerbatimInput[
			frame = lines, fontsize = {\footnotesize}, tabsize = 2,
			baselinestretch = 1, firstline = 25, numbers = left
		]{pkuthss.cls}

		\subsection{\texttt{pkuthss-utf8.def} 和 \texttt{pkuthss-gbk.def}}
		\VerbatimInput[
			frame = lines, fontsize = {\footnotesize}, tabsize = 2,
			baselinestretch = 1, firstline = 25, numbers = left
		]{pkuthss-utf8.def}

		\subsection{\texttt{pkuthss-extra.sty}}
		\VerbatimInput[
			frame = lines, fontsize = {\footnotesize}, tabsize = 2,
			baselinestretch = 1, firstline = 25, numbers = left
		]{pkuthss-extra.sty}

	\section{pkuthss 说明(示例)文档的源代码}

	本文档的源代码中大部分已经有了较为详细的注释,
	故请直接参照相应文件中的注释。

	\myemph{%
		注:%
		\texttt{img/} 目录中的 \texttt{Makefile} 和%
		两个 PostScript(\texttt{.eps})文件(都是文本文件)中%
		也有详细的注释哦 :)%
	}

\flushbottom % 取消 \raggedbottom 的作用。


	% vim:ts=4:sw=4
%
% Copyright (c) 2008-2009 solvethis
% Copyright (c) 2010-2012 Casper Ti. Vector
% Public domain.

\chapter{更新记录}
\raggedbottom % 避免某些奇怪的“Underfull \vbox”警告。

\section{1.3 版以后的更新记录}
\VerbatimInput[
	tabsize = 4, fontsize = {\small}, baselinestretch = 1
]{./verbatim/ChangeLog.txt}

\section{1.3 及其以前版本的更新记录}
\VerbatimInput[
	tabsize = 4, fontsize = {\small}, baselinestretch = 1.1
]{./verbatim/ChangeLog-upto-1.3.txt}

\flushbottom % 取消 \raggedbottom 的作用。



	% 以下为正文之后的部分。
	\backmatter

	% 致谢。
	% vim:ts=4:sw=4
%
% Copyright (c) 2008-2009 solvethis
% Copyright (c) 2010-2012 Casper Ti. Vector
% Public domain.

\chapter{致谢}

感谢北大未名 BBS 的 MathTools 版和 Thesis 版诸位同学的支持。
特别感谢 pkuthss 模版的最初创作者 solvethis 网友,
以及不断地对 Casper 提出的诸多问题予以解答的 cauchy 网友 :)


\end{document}

